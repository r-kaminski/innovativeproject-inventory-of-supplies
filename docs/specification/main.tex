\documentclass{article}
\usepackage[utf8]{inputenc}
\usepackage[T1]{fontenc}

\title{Projekt Zespołowy Wymagania  Funkcjonalne}
\author{ }
\date{March 2019}

\begin{document}

\maketitle

\section{Introduction}
\begin{enumerate}
\item System zawiera katalog przedmiotów, do którego można wstawiać nowe produkty, edytować lub usuwać już umieszczone w katalogu.
\begin{itemize}
\item Opis: Utworzona zostanie baza danych zawierająca produkty i użytkownik będzie mógł operować na własnościach tych obiektów o ile będzie miał odpowiednie uprawnienia.
\item Pochodzenie funkcjonalności: Specyfikacja wysłana przez zleceniodawcę. 
\item Kryterium satysfakcji: Zadanie zostanie w całości wykonane.
\item Wymóg: Obowiązkowe
\end{itemize}
\item Wyszukiwanie przedmiotów umieszczonych w katalogu: po nazwie, kategorii i kodzie QR.
\begin{itemize}
\item Opis:Stworzenie opcji przeszukiwania produktów z bazy danych z poziomu aplikacji użytkownika. 
\item Pochodzenie funkcjonalności: Specyfikacja wysłana przez zleceniodawcę. 
\item Kryterium satysfakcji: Utworzenie trzech działających implementacji wyszukiwania. 
\item Wymóg: Obowiązkowe
\end{itemize}
\item Drukowanie kodów QR znajdujących się w katalogu.
\begin{itemize}
\item Opis: Wydruk kodu QR przypisanego do przedmiotu znajdującego się w bazie danych. 
\item Pochodzenie funkcjonalności: Specyfikacja wysłana przez zleceniodawcę. 
\item Kryterium satysfakcji: Zadanie w całości wykonane.
\item Wymóg: Obowiązkowe
\end{itemize}
\item Skanowanie i dodawanie nowych kodów QR do przedmiotów z katalogu.
\begin{itemize}
\item Opis: Skanowanie przy pomocy kamery telefonu, lub skanera podłączonego do komputera i dodanie uzyskanego skanu do przedmiotu znajdującego się w bazie danych.
\item Pochodzenie funkcjonalności: Specyfikacja wysłana przez zleceniodawcę. 
\item Kryterium satysfakcji: Zadanie w całości wykonane.
\item Wymóg: Obowiązkowe
\end{itemize}
\item  Zgłaszanie usterek przedmiotów z katalogu, lub braku materiałów.
\begin{itemize}
\item Opis: Dodanie inforamcji do przedmiotu znajdującego się w bazie danych o jego stanie technicznym, lub poinformowanie innych użytkowników o niskim stanie zasobów.
\item Pochodzenie funkcjonalności: zespół projektowy.
\item Kryterium satysfakcji: Zadanie w całości wykonane.
\item Wymóg: Opcjonalne.
\end{itemize}
\item  Rezerwacja stanowisk dostępnych w pracowni.
\begin{itemize}
\item Opis: Możliwość zgłaszania zainteresowania jednym z dostępnych stanowisk i zarezerwowanie wspomnianego stanowiska w określonych godzinach. Inni użytkownicy będą mogli wyświetlić dostępność stanowiska w dowolnym momencie.
\item Funkcjonalność dodana przez: zespół projektowy.
\item Kryterium satysfakcji: Zadanie w całości wykonane.
\item Wymóg: Opcjonalne.
\end{itemize}
\item Lokalizowanie przedmiotów i stanowisk na mapie.
\begin{itemize}
\item Opis: Nałożenie na mapę pracowni wszystkich dostępnych urzadzeń i stanowisk i wyświetlanie tych informacji użytkownikowi, żeby wiedział gdzie ma dane narzędzie odłożyć po zakończonej pracy.
\item Funkcjonalność dodana przez: Specyfikacja wysłana przez zleceniodawcę.
\item Kryterium satysfakcji: Zadanie w całości wykonane.
\item Wymóg: Obowiązkowe
\end{itemize}
\item Poradnik dla użytkownika jak korzystać z systemu.
\begin{itemize}
\item Opis: Przygotowana wcześniej instrukacja jak korzystać z aplikacji i strony internetowej. Przedstawienie funkcjonalności w czytelny sposób.
\item Funkcjonalność dodana przez: zespół projektowy.
\item Kryterium satysfakcji: Zadanie w całości wykonane.
\item Wymóg: Opcjonalne.
\end{itemize}
\item Podział na pokoje.
\begin{itemize}
\item Opis: Przydzielenie użytkowników do mniejszych przestrzeni roboczych. 
\item Funkcjonalność dodana przez: zespół projektowy.
\item Kryterium satysfakcji: Zadanie w całości wykonane.
\item Wymóg: Opcjonalne.
\end{itemize}
\item Tworzenie pokoju i zarządzanie jego zawartością.
\begin{itemize}
\item Opis: Utworzenie instancji pokoju, który połączony jest z bazą dostępnych urządzeń i stanowisk. Użytkownicy przypisani są do odpowiednich pokoi i mają dostęp do tych zasobów.
\item Funkcjonalność dodana przez: zespół projektowy.
\item Kryterium satysfakcji: Zadanie w całości wykonane.
\item Wymóg: Opcjonalne.
\end{itemize}
\item Możliwość dołączenia do istniejącego pokoju.
\begin{itemize}
\item Opis: Użytkownik może zostać przypisany do istniejącego pokoju.
\item Funkcjonalność dodana przez: zespół projektowy.
\item Kryterium satysfakcji: Zadanie w całośći wykonane.
\item Wymóg: Opcjonalne.
\end{itemize}
\item Wykonanie inwentaryzacji przedmiotów znajdujących się w pokoju.
\begin{itemize}
\item Opis: Zestawienie przedmiotów znajdujących się w bazie danych przypisanej do pokoju wraz z ich stanem technicznym.
\item Funkcjonalność dodana przez: Specyfikacja wysłana przez zleceniodawcę.
\item Kryterium satysfakcji: Zadanie w całości wykonane.
\item Wymóg: Obowiązkowe.
\end{itemize}
\item Wygenerowanie raportu z inwentaryzacji.
\begin{itemize}
\item Opis: Stworzenie dokumentu z przeprowadzonej inwentaryzacji pokoju. Dokument ten będzie miał określoną strukturę opisaną przez zleceniodawcę.
\item Funkcjonalność dodana przez: Specyfikacja wysłana przez zleceniodawcę.
\item Kryterium satysfakcji: Zadanie w całości wykonane.
\item Wymóg: Obowiązkowe.
\end{itemize}
\item Przeglądanie opisu dowolnego przedmiotu z katalogu.
\begin{itemize}
\item Opis: Użytkownicy mają możliwość przeglądania dokładnego opisu przedmiotów znajdujących się w katalogu.
\item Funkcjonalność dodana przez: Specyfikacja wysłana przez zleceniodawcę.
\item Kryterium satysfakcji: Zadanie w całości wykonane.
\item Wymóg: Obowiązkowe.
\end{itemize}
\item Komentarze
\begin{itemize}
\item Opis: Dodawanie komentarzy pod produktami z katalogu.
\item Funkcjonalność dodana przez: Specyfikacja wysłana przez zleceniodawcę.
\item Kryterium satysfakcji: Zadanie w całości wykonane.
\item Wymóg: Opcjonalne
\end{itemize}
\item Zdjęcia narzędzi.
\begin{itemize}
\item Opis: Dodawanie zdjęć do przedmiotów. Zdjęcia powinny być hostowane przez zewnętrzny serwer.
\item Funkcjonalność dodana przez: Specyfikacja wysłana przez zleceniodawcę.
\item Kryterium satysfakcji: Zadanie w całości wykonane.
\item Wymóg: Opcjonalne
\end{itemize}
\item System pozwala na wykonanie backupu.
\begin{itemize}
\item Opis: Możliwość stworzenia kopii zapasowej bazy danych
\item Funkcjonalność dodana przez: Specyfikacja wysłana przez zleceniodawcę.
\item Kryterium satysfakcji: Zadanie w całości wykonane.
\item Wymóg: Opcjonalne
\end{itemize}
\item Generowanie zestawienia w pliku arkusza kalkulacyjnego
\begin{itemize}
\item Opis: Możliwość wyeksportowania danych z aplikacji do arkusza w formacie .csv
\item Funkcjonalność dodana przez: Specyfikacja wysłana przez zleceniodawcę.
\item Kryterium satysfakcji: Zadanie w całości wykonane.
\item Wymóg: Opcjonalne
\end{itemize}
\item Importowanie zestawienia z pliku arkusza kalkulacyjnego
\begin{itemize}
\item Opis: Importowanie danych do aplikacji z arkusza w formacie .csv
\item Funkcjonalność dodana przez: Specyfikacja wysłana przez zleceniodawcę.
\item Kryterium satysfakcji: Zadanie w całości wykonane.
\item Wymóg: Opcjonalne
\end{itemize}
\item Podział na klasy użytkowników.
\begin{itemize}
\item Opis: System ma wspierać role użytkowników, którzy mogą mieć różne uprawnienia.
\item Funkcjonalność dodana przez: Specyfikacja wysłana przez zleceniodawcę.
\item Kryterium satysfakcji: Wprowadzenie roli użytkownika i super użytkownika
\item Wymóg: Obowiązkowe
\end{itemize}
\item Logowanie i rejestrowanie użytkownika do systemu.
\begin{itemize}
\item Opis: Użytkownik ma możliwość zalogowania się na swoje konto, oraz stworzenia nowego konta.
\item Funkcjonalność dodana przez: Specyfikacja wysłana przez zleceniodawcę.
\item Kryterium satysfakcji: Zadanie w całości wykonane. Logowanie zrealizowane w sposób spełniający wymagania bezpieczeństwa.
\item Wymóg: Obowiązkowe
\end{itemize}
\end{enumerate}
\end{document}
